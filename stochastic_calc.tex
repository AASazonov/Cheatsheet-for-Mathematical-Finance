\part{Stochastic Calculus}

    \section{Girsanov Theorem and Novikov Condition}
        Let us take a filtered probability space $\left(\Omega, \mathcal{F}, (\mathcal{F}_t)_{t=0}^T, \P\right)$, and a Brownian motion $W_t$ defined on it.
        \begin{theorem}
            (Girsanov)
            \emph{
                Let $\mu_t$ be a predictable process such that $\int_0^T \mu_t^2 dt < \infty$ almost surely. Define a process\begin{equation}
                    Z_t = exp\left(-\int_0^t \mu_s dW_s - \frac{1}{2}\int_0^t \mu_s^2 ds\right) 
                \end{equation}
                If $Z_t$ is a martingale, then 
                \begin{equation}
                    \tilde{W}_t = W_t + \int_0^t \mu_s ds
                \end{equation}
                is a Brownian motion w.r.t. measure $Q$ such that $dQ = Z_t dP$.
            }   
        \end{theorem}

        \begin{theorem}
            (Girsanov)
            \emph{
                Let $W_t$ be a $d$-dimensional Brownian motion with non-correlated components, $\mu_t$ be a $d$-dimensional predictable process such that $\int_0^T \left\|\mu_t\right\|^2 dt < \infty$ almost surely. Define a process
                \begin{equation}
                    Z_t = exp\left(-\int_0^t \mu_s \cdot dW_s - \frac{1}{2}\int_0^t \left\|\mu_s\right\|^2 ds\right) 
                \end{equation}
                If $Z_t$ is a martingale, then 
                \begin{equation}
                    \tilde{W}_t = W_t + \int_0^t \mu_s ds
                \end{equation}
                is a Brownian motion with non-correlated components w.r.t. measure $Q$ such that $dQ = Z_t dP$.
            }   
        \end{theorem}

        When $Z_t$ is a martingale? One of the easiest ways to answer this question is to remember the 

        \begin{theorem} (Novikov condition)
            \emph{
                Let \begin{equation*}
                    \E \left[exp\left(\frac{1}{2}\int_0^T \left\|\mu_t\right\|^2 dt\right)\right] < \infty.
                \end{equation*}
                Then
                \begin{equation}
                    Z_t = exp\left(-\int_0^t \mu_s \cdot dW_s - \frac{1}{2}\int_0^t \left\|\mu_s\right\|^2 ds\right) 
                \end{equation}
                is a martingale.
            }
        \end{theorem}

        \begin{theorem}
            \emph{
                If $Q \sim \P$, $dQ = Z_t dP$, then for any $\mathcal{F}_T$-measurable r.v. $X$
                \begin{equation}
                    \E^Q\left[X \vert \mathcal{F}_t\right] = \frac{1}{Z_t}\E^\P\left[Z_T X \vert \mathcal{F}_t\right]
                \end{equation}
            }
        \end{theorem}

    \section{Weak and Strong Solutions of a SDE}
        Assume a one-dimensional SDE
        \begin{equation}\label{SDE1}
            dX_t = b(t, X_t) dt + \sigma(t, X_t) dW_t, \qquad X_0 = x
        \end{equation}

        \begin{definition}
            A \emph{weak} solution of equation \eqref{SDE1} is a pair of processes $(X_t, W_t)$ on some probability space with filtration $(\mathcal{F}_t)_{t\geq0}$ such that $W_t$ is a Brownian motion w.r.t. $\mathcal{F}_t$ and 
            \begin{equation}
                X_t = x+ \int_0^t b(s, X_s) ds + \int_0^t \sigma(s, X_s) dW_s,
            \end{equation}
            and both integrals exist for all $t \leq T$
        \end{definition}

        \begin{definition}
            A \emph{strong} solution of equation \eqref{SDE1} is a weak solution $(X_t, W_t)$, where $X_t$ is adapted to filtration generated by $W_t$.
        \end{definition}

        \begin{definition}
            Equation \eqref{SDE1} has a \emph{weak uniqueness} of a solution, if for any two weak solutions, $X_t$ and $\tilde X_t$ are equal in distribution.
        \end{definition}

        \begin{definition}
            Equation \eqref{SDE1} has a \emph{strong uniqueness} of a solution, if for any two strong solutions, $X_t$ and $\tilde X_t$ are equal a.s.
        \end{definition}

        \begin{theorem}(It\^o)
            \emph{
                Let there exist $C> 0$ such that
                \begin{align}
                    |b(t, x) - b(t, y)| + |\sigma(t, x) - \sigma(t, y)| &< C|x-y| \\
                    |b(t, x)|+|\sigma(t, x)| &< C(1+|x|)
                \end{align}
                Then there exists a strong unique solution for the equation \eqref{SDE1}.
            }
        \end{theorem}

        \begin{theorem}(Zvonkin)
            \emph{
                Let \eqref{SDE1} satisfy the following conditions:
                \begin{enumerate}
                    \item $b$ is bounded
                    \item $\sigma$ is continuous
                    \item $|\sigma(t, x) - \sigma(t, y)| < C\sqrt{|x-y|}$, $|\sigma(t, x)| \geq \epsilon > 0 $
                \end{enumerate}
                Then there exists a strong unique solution for the equation \eqref{SDE1}.
            }
        \end{theorem}

        \begin{theorem}(Skorokhod)
            \emph{
                Let $b(t, x)$, $\sigma(t, x)$ be bounded and continuous
                Then there exists a weak unique solution for the equation \eqref{SDE1}.
            }
        \end{theorem}

        \begin{theorem}(Struk, Varadan)
            \emph{
                Let \eqref{SDE1} satisfy the following conditions:
                \begin{enumerate}
                    \item $b$ is bounded
                    \item $\sigma$ is continuous
                    \item $\sigma(t, x) \neq 0 $
                \end{enumerate}
                Then there exists a weak unique solution for the equation \eqref{SDE1}.
            }
        \end{theorem}
        \begin{definition}
            A SDE is called \emph{homogeneous} if it has the form of
            \begin{equation}\label{SDEh}
                dX_t = b(X_t) dt + \sigma(X_t) dW_t
            \end{equation}
        \end{definition}

        \begin{definition}
            For a homogeneous SDE we define the following functions:
            \begin{equation}
                \rho(x):= exp\left(- \int \frac{2b(x)}{\sigma^2(x)}dx\right) \quad s(x) = \int \rho(x)dx
            \end{equation}
            The function $s(x)$ is called a \emph{scale}. 
        \end{definition}

        \begin{nb}
            Using the It\^o's formula one can show that if $\forall x \ \ s(x) < \infty$, then for any weak solution $X_t$, $s(X_t)$ is a local martingale. Note that $\rho(x)$ is defined up to multiplication by a positive scalar, and $s(x)$ is defined up to an affine transformation.
        \end{nb}

        \begin{theorem}(Engelbert, Schmidt)
            \emph{
                Let \eqref{SDEh} satisfy the following conditions:
                \begin{enumerate}
                    \item $\sigma(t, x) \neq 0 $
                    \item $\int_K \frac{1+|b(y)|}{\sigma^2(y)}dy < \infty$ for any compact set $K \subseteq \mathbb{R}$ 
                    \item one of the following conditions stand:
                    \begin{equation}
                        \int_\mathbb{R} \rho(x)dx < \infty \text{   or   } \int_\mathbb{R} \rho(x)dx = \infty, \quad \int_\mathbb{R} \frac{|s(x)|}{\rho(x) \sigma^2(x)}dx = \infty
                    \end{equation}
                \end{enumerate}
                Then there exists a weak unique solution for the equation \eqref{SDEh}.
            }
        \end{theorem}
        
    \section{Classification of Singular Points of a SDE}
        In this section we assume equation \eqref{SDEh} with $\sigma(x)\neq 0 \ \ \forall x \in \mathbb{R}$.
        \begin{definition}
            A point $y\in \mathbb{R}$ is called a \emph{singular} point of equation \eqref{SDEh}, if\begin{equation}
                \forall \epsilon > 0 \qquad \int_{y-\epsilon}^{y+\epsilon} \frac{1+|b(x)|}{\sigma^2(x)}dx = \infty
            \end{equation}
        \end{definition}
        Further we assume $x = 0$ to be the only singular point of a SDE \eqref{SDEh}. For some $a > 0$ define
        \begin{equation}
            \rho(x):= exp\left(- \int_x^a \frac{2b(y)}{\sigma^2(y)}dy\right)
        \end{equation}
        and
        \begin{equation}
            s(x) = \left\{\begin{aligned}
                \int_0^x \rho(y)dy, &\text{ if } \int_0^a \rho(y)dy < \infty \\ 
                -\int_x^a \rho(y)dy, &\text{ if } \int_0^a \rho(y)dy = \infty
            \end{aligned}
            \right.
        \end{equation}

        \begin{table}[h]
            \begin{tabular}{| c | l |}
                \hline
                Type  & Conditions                                                                                                                                                      \\
                \hline
                0     & $\frac{1+|b(y)|}{\sigma^2(y)}dy < \infty$                                                                                                                       \\
                1     & $\int_0^a \rho(y)dy < \infty, \quad \int_0^a \frac{1+|b(y)|}{\rho(y)\sigma^2(y)}dy = \infty, \quad \int_0^a \frac{1+|b(y)|}{\rho(y)\sigma^2(y)}s(y)dy < \infty$ \\
                2     & $\int_0^a \rho(y)dy < \infty, \quad \int_0^a \frac{1+|b(y)|}{\rho(y)\sigma^2(y)}dy < \infty, \quad  \int_0^a \frac{|b(y)|}{\sigma^2(y)}dy = \infty$             \\
                3     & $\int_0^a \rho(y)dy = \infty, \quad \int_0^a \frac{1+|b(y)|}{\rho(y)\sigma^2(y)}|s(y)|dy < \infty$                                                              \\
                4     & $\int_0^a \rho(y)dy < \infty, \quad \int_0^a \frac{s(y)}{\rho(y)\sigma^2(y)}dy = \infty $                                                                       \\
                5     & $\int_0^a \rho(y)dy = \infty, \quad \int_0^a \frac{1+|b(y)|}{\rho(y)\sigma^2(y)}|s(y)|dy = \infty$                                                              \\
                6     & $\int_0^a \rho(y)dy < \infty, \quad \int_0^a \frac{1+|b(y)|}{\rho(y)\sigma^2(y)}dy = \infty, \quad \int_0^a \frac{|s(y)|}{\rho(y)\sigma^2(y)}dy < \infty$           \\
                \hline
            \end{tabular}
            \caption{Classification of Finite Singular Points, Right Type}
        \end{table}
        For some $a > 0, x \geq a$ define
        \begin{equation}
            \rho(x):= exp\left(- \int_a^x \frac{2b(y)}{\sigma^2(y)}dy\right), \qquad s(x) = -\int_x^\infty \rho(y)dy
        \end{equation}

        \begin{table}[h]
            \begin{tabular}{| c | l |}
            \hline
            Type            & Conditions \\
            \hline
            A               &            \\
            B               &            \\
            C               &            \\
            \hline
            \end{tabular}
            \caption{Classification of $+\infty$}
        \end{table}




    \section{PDEs and SDEs}
        \begin{theorem}(Feynman-Kac)
            \emph{
                Consider the following PDE
                \begin{multline}
                    \frac{\partial u}{\partial t}(x,t) + \mu(x,t) \frac{\partial u}{\partial x}(x,t) + \tfrac{1}{2} 
                    \sigma^2(x,t) \frac{\partial^2 u}{\partial x^2}(x,t) - \\ -V(x,t) u(x,t) + f(x,t) = 0, 
                \end{multline}
                defined for all $x \in \mathbb{R}$ and $t \in [0, T]$, subject to the terminal condition
                $u(x,T)=\psi(x)$.
                Then the solution can be written as a conditional expectation
                \begin{equation*}
                    u(x,t) = \E^Q\left[\left. \int_t^T e^{-  \int_t^r V(X_\tau,\tau) d\tau}f(X_r,r)dr + e^{-\int_t^T V(X_\tau,\tau) d\tau}\psi(X_T) \right| X_t=x \right] 
                \end{equation*}
                under the probability measure $Q$ such that $X$ is an It\^o process driven by the equation
                \begin{equation}
                    dX = \mu(X,t)dt + \sigma(X,t)dW^Q,
                \end{equation}
                with $W^Q(t)$ is a Wiener process under $Q$, and the initial condition for $X_t$ is $X_t = x$.
            }  
        \end{theorem}